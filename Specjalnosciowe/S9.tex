\section{S9 -- Zasady projektowania bezpiecznych systemów i sieci komputerowych}

Podstawą bezpieczeństwa systemu informatycznego jest dobrze opracowany projekt, wdrożony z użyciem właściwie dobranych technologii renomowanych producentÛw, zarządzany przez wykwalifikowaną kadrę informatyczną. Projektowane zabezpieczenia powinny być oparte w znacznej mierze na wynikach specyfikacji wymagań bezpieczeństwa, a także ogólnej teorii zabezpieczeń (m.in. wymagane jest dokonanie weryfikacji odporności systemu na strategię włamań Island Hopping Attack). Tworzenie zabezpieczeń systemu informatycznego powinno odbywać się w przemyślany, wcześniej szczegółowo zaplanowany sposób zgodnie ze sprawdzoną metodyką. 

Podczas projektowania bezpiecznych sieci i systemów należy pamiętać, by używać \textbf{jedynie wymaganego} sprzętu i oprogramowanie. Każda nadmiarowa rzecz może powodować istotne luki w bezpieczeństwie systemu. Podczas nadawania \textbf{uprawnień} trzeba dawać ich \textbf{tylko tyle, ile jest faktycznie potrzebne}. Sieć powinna być zaprojektowana w ten sposób, aby maksymalnie \textbf{ograniczyć możliwość podłączania niepotrzebnych i obcych urządzeń}. Istotnym elementem jest także uwzględnienie podczas projektowania mechanizmów monitorujących i sterujących siecią – oczywiście muszę one spełniać wymienione powyżej warunki. Bezpieczny system komputerowy również musi spełniać podobne warunki. Procesy muszą działać zawsze możliwie \textbf{z najniższymi możliwymi uprawnieniami. Udostępniane} powinny być \textbf{jedynie naprawdę wymagana elementy systemu} (np. baza danych postawiona na serwerze WWW wcale nie musi być bezpośrednio dostępna dla klientów z zewnątrz – wystarczy, że ma do niej dostęp usługa HTTP). Ponadto mechanizmy ochronne muszę być \textbf{możliwie proste, bez cudowania} i zaimplementowane w możliwie \textbf{niskich warstwach systemu}. Ważnym – chyba często pomijanym – aspektem jest taka budowa systemu bezpieczeństwa, \textbf{aby nie drażnił on i nie zniechęcał użytkowników.}