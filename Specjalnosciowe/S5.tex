\section{S5 -- Charakterystyka wybranej techniki wirtualizacji}

\textbf{Wirtualizacja} jest to metodologia polegająca na dzieleniu zasobów komputera na wiele środowisk. Działają one poprzez zastosowanie jednej lub wielu technologii, np. sprzętowe lub softwarowe partycjonowanie, tworzenie maszyny wirtualnej, przydzielanie czasu.

\begin{figure}[H]
	\centering
	\includegraphics[scale=0.45]{s5_wirtualizacja.png}
	\caption{Wirtualizacja serwerów}
\end{figure}


\begin{itemize}
	\setlength\itemsep{1pt}
	\item \textbf{Obszary wirtualizacji:}
	\item \textbf{Wirtualizacja serwerów} (ang. \textit{Server Virtualization}) – technologia ta  umożliwia wielu aplikacjom działanie na wielu systemach operacyjnych uruchomionych na tym samym fizycznym serwerze z wykorzystaniem wolnej mocy. Dzięki temu można w pełni wykorzystać moc obliczeniową i zasobową serwerów, stawiając kolejne maszyny wirtualne na jednym fizycznym serwerze – zamiast kupować kolejne serwery fizyczne. Dodatkową zaletą jest też szybsze dostarczenie usług dla biznesu w ciągu bardzo krótkiego czasu tworzenia nowego wirtualnego systemu operacyjnego gotowego do pracy. Ponadto podstawowe zadania konserwacyjne związane z serwerami są znacznie uproszczone i przyśpieszone.
	\item \textbf{Wirtualizacja stacji roboczych} (utrzymywana na kliencie) (ang. \textit{Client-HostedDesktop Virtualization}) – to rodzaj wirtualizacji, która oddziela system operacyjny kliencki od sprzętu fizycznego i umożliwia uruchomienia maszyn wirtualnych na jednym komputerze PC obok klienckiego systemu operacyjnego hosta. Może być centralnie zarządzana, a obrazy maszyn wirtualnych mogą być dostarczane również centralnie. Technologię tę stosuje się wtedy, gdy zachodzi potrzeba zapewnienia zgodności aplikacji z systemem operacyjnym. Aplikacje są zainstalowane w maszynie wirtualnej i tam wykonuje się ich kod. Dostarczenie aplikacji do użytkownika odbywa się najczęściej w trybie seamless, czyli bez dodatkowej otoczki pulpitu – jedynie widok samej aplikacji.
	\item \textbf{Wirtualizacja stacji roboczych} (utrzymywana na serwerze)(ang. \textit{Server-Based Desktop Virtualization}) – znana bardziej pod nazwą \textbf{Wirtualna infrastruktura stacji roboczych} (ang. \textit{Virtual Desktop Infrastructure – VDI}). Jest to przeniesienie klasycznego środowiska pracy użytkownika, opartego na stacjach roboczych, do centrum przetwarzania danych, gdzie stacja robocza jest uruchomiona w postaci maszyny wirtualnej, a dostęp do niej zapewniony jest drogą sieciową. Połączenie może być zrealizowane z dowolnego komputera PC, laptopa lub cienkiego klienta, który jest preferowaną formą dostępu do VDI. Sam w sobie VDI nie jest jedną technologią – to połączenie kilku niezależnych rozwiązań, które w podstawowej architekturze łączy ze sobą wirtualizację serwerów z wirtualizacją prezentacji.
	\item \textbf{Wirtualizacja sieci} (ang. \textit{Network Virtualization}), którą można podzielić na dwa rodzaje – zewnętrzną oraz wewnętrzną. Zewnętrzna wirtualizacja sieci to znana od lat technologia VLAN (802.1Q), czyli logiczny podział segmentów sieci na fizycznym sprzęcie sieciowym przez znakowanie ramek. Wewnętrzna natomiast to rozwiązanie polegające na tworzeniu wirtualnych przełączników i portów na poziomie hiperwizora. Takie zwirtualizowane przełączniki dostarczają komunikację sieciową dla maszyn wirtualnych oraz łączą je z zewnętrzną fizyczną siecią.
	\item \textbf{Chmura prywatna} (ang. \textit{Private Cloud}) to realizacja koncepcji chmury wewnątrz własnej firmy, czyli na własnych serwerach i własnym oprogramowaniu. W wersji podstawowej w chmurze prywatnej można zrealizować model infrastruktury jako usługi (ang. Infrastructure as a Service  – IaaS). Bardziej złożone chmury prywatne mogą realizować model platformy jako usługi (ang. Platform as a Service – PaaS). Odbiorca mocy obliczeniowej lub zasobów nie wie tak naprawdę, w którym miejscu infrastruktury firmy się znajduje, gdyż jego przestrzeń może dynamicznie wędrować między serwerami oraz dynamicznie się skalować. Podsumowując, chmura prywatna to nic innego, jak usługi przetwarzania w chmurze świadczone przez dział IT dla własnej firmy.
\end{itemize}

\begin{itemize}
	\setlength\itemsep{1pt}
	\item[] \textbf{Zalety Wirtualizacji: }
	\item redukcja kosztów w aspekcie długoterminowym,
	\item ograniczenie rozrostu serwerowni,
	\item konsolidacja zasobów,
	\item elastyczność przydziału zasobów,
	\item łatwe testowanie nowych aplikacji oraz sieci komputerowych,
	\item możliwość uruchomienia większej ilości usług na jednym sprzęcie.
	\item korzystanie ze starszych aplikacji, niekompatybilnych z obecnie produkowanym sprzętem czy nowymi systemami,
	\item dostępność na bardzo wysokim poziomie.
\end{itemize}
	
\begin{itemize}
	\setlength\itemsep{1pt}
	\item[] \textbf{Wady: }
	\item Infrastruktura informatyczna, musi być bardzo wydajna i niezawodna,
	\item koszt inwestycji w mocne serwery,
	\item zakup licencji,
	\item wirtualne maszyny mogą być w niektórych przypadkach mniej wydajne od tradycyjnych rozwiązań,
	\item potrzeba wykwalifikowanej kadry zarządzającej.
\end{itemize}
	
My podczas zajęć korzystaliźmy z wirtualizacji VirtualBoxa, emulując np. Windowsa 7, czy CentOS-a.
		
		
		
		
		
		
		
