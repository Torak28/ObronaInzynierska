\section{S2 -- Protokoły rozległych sieci komputerowych.}

\textbf{Sieć WAN} (z ang. \textit{Wide Area Network}) – sieć komputerowa znajdująca się na obszarze wykraczającym poza miasto, kraj, kontynent.

\begin{itemize}
	\setlength\itemsep{1pt}
	\item Łączą ze sobą urządzenia rozmieszczone na dużych obszarach geograficznych (np. kraju, kontynentu).
	\item W celu zestawienia łącza lub połączenia między dwoma miejscami korzystają z usług operatorów telekomunikacyjnych, np. Netia, TP S.A., NASK, Exatel.
	\item Wykorzystują różne odmiany transmisji szeregowej.
\end{itemize}

Sieć WAN działa w warstwie fizycznej oraz warstwie łącza danych modelu odniesienia OSI. Łączy ona ze sobą sieci lokalne, które są zazwyczaj rozproszone na dużych obszarach geograficznych. Sieci WAN umożliwiają wymianę ramek i pakietów danych pomiędzy routerami i przełącznikami oraz obsługiwanymi sieciami LAN.

Protokoły sieci rozległych można je podzielić na następujące grupy technik łączenia:

\begin{itemize}
	\setlength\itemsep{1pt}
	\item \textbf{komutacja kanałów} - polega na przydzieleniu wybranemu połączeniu wybranej sekwencji połączonych kanałów od terminala źródłowego do terminala docelowego. W sieciach z komutacją kanałów przesyłanie danych następuje dopiero po ustanowieniu połączenia, czyli uzyskaniu specjalnej trasy pomiędzy systemem nadawcy a systemem odbiorcy. Trasa jest sekwencją kolejno połączonych kanałów. Kanały te zostają zajęte przez cały czas, w którym trwa połączenie. Zarezerwowane kanały nie mogą być używane przez inne połączenia. Samo przesyłanie informacji odbywa się w 3 fazach: ustanowienie połączenia, transfer danych, rozłączenie połączenia.
	
	Jedną z wad techniki komutacji kanałów jest konieczność ustanowienia połączenia między użytkownikami zanim zostaną przesłane dane. Powoduje to powstawanie opóźnień w przesyłaniu informacji oraz to, że kanały są niewykorzystywane podczas zestawiana połączenia, rozłączania połączenia oraz w przerwach między transmisją przy zestawionym połączeniu. W efekcie wzrasta koszt utrzymania takiej sieci a spada jej efektywność. Na plus można zaliczyć wysoką jakość transmisji poprzez trwały kanał (tzn. ustanowiony wcześniej kanał o niezmiennych parametrach). Technika ta wykorzystywana jest w sieciach gdzie przesyła się głos i dane.
	
	\item \textbf{łącza dzierżawione} - łącze telekomunikacyjne (miedziane lub światłowód) wydzierżawione od innego operatora lub osoby indywidualnej po to, aby móc przesłać po nim swój sygnał. Łącze dzierżawione transmisji danych ma stałą przepustowość.
	
	Opłaty za dzierżawę łącza uiszczane są niezależnie od stopnia jego wykorzystania. Na przykład niezależnie od ilości przesyłanych danych i czasu transmisji w ustalonym okresie, którego dotyczy opłata.
	
	Łącza tego typu wykorzystywane są w sytuacjach wymagających zagwarantowanej dyspozycyjności, na przykład dostępu do internetu o gwarantowanej przepustowości. Umożliwiają one klientom uruchamianie własnych usług internetowych dostępnych w trybie całodobowym.
	
	\item \textbf{komutacja komórek} - jest odmianą komutacji pakietów, w której pakiety zastąpiono krótkimi komórkami o stałej długości, co pozwala na sprzętową realizacje komutacji. Z komutacją łączy wiąże komutacje ATM konieczność zestawiania połączenia między stacjami końcowymi, przed rozpoczęciem przesyłania informacji użytkowej, tzn. zapewnienia (w sensie statycznym) dostępności zasobów sieci (łączy, buforów w węzłach) w czasie przesyłania tej informacji. W komutacji komórek ATM są aktualizowane wartości wirtualnych ścieżek i kanałów (VPI/VCI).
	
	\item \textbf{komutacja pakietów} - technika komutacji pakietów należy do najbardziej elastycznych technik komutacji stosowanych we współczesnych sieciach. Polega ona na przesyłaniu danych przez sieć w postaci pakietów. W odróżnieniu od techniki komutacji łączy pozwala użytkownikom nawiązywać połączenia z wieloma innymi użytkownikami jednocześnie.
	
	Pakiety powstają w wyniku podzielenia informacji użytkownika na części o stałej długości (wyjątkiem jest ostatnia część, której długość może być mniejsza) i opatrzenie tych „wycinków” w nagłówek N, też o stałej długości. Nagłówek zawiera informacje, które umożliwiają pakietowi dojście z punktu źródłowego do docelowego, węzłom sieci sprawdzenie poprawności zawartych w pakiecie danych, a punktom docelowym właściwie zestawić i odtworzyć podzieloną informację. W nagłówku zatem, w zależności od organizacji sieci, znajdują się następujące informacje: adresy źródłowy i docelowy, numer portu źródłowego i docelowego, numer pakietu, wskaźnik ostatniego pakietu, a także identyfikator zawartej w pakiecie informacji.
	
	Efektem tego jest kilka cech komutacji pakietów:
	
	\begin{itemize}
		\setlength\itemsep{1pt}
		\item odporność na uszkodzenia sieci (uszkodzone urządzenia są po prostu omijane)
		\item możliwość docierania pakietów w przypadkowej kolejności (ze względu na różne ścieżki transmisji)
		\item opóźnienia związane z buforowaniem pakietów w routerach
		\item duża przepustowość efektywna sieci
		\item Alternatywą dla komutacji pakietów jest komutacja łączy.
	\end{itemize}
	
	Przykłady sieci wykorzystujących komutację pakietów:
	
	\begin{itemize}
		\setlength\itemsep{1pt}
		\item IPv4
		\item IPv6
	\end{itemize}
	
\end{itemize}

\textbf{Przykładowe protokoły:}
\begin{itemize}
	\setlength\itemsep{1pt}
	\item \textbf{komutacja kanałów:}
	\begin{itemize}
		\setlength\itemsep{1pt}
		\item \textbf{ISDN} - technologia sieci telekomunikacyjnych mająca na celu wykorzystanie infrastruktury PSTN do bezpośredniego udostępnienia usług cyfrowych użytkownikom końcowym (bez pośrednictwa urządzeń analogowych) (ang. end-to-end circuit-switched digital services). Połączenia ISDN zalicza się do grupy połączeń wdzwanianych (komutowanych).
		
		ISDN jest znormalizowana w zaleceniach ITU-T oraz standardach ETSI. Europejskie zostały wprowadzone do systemu Polskich Norm jako grupa ICS 33.080 - Sieć Cyfrowa z Integracją Usług
		\item \textbf{PPP} - protokół warstwy łącza danych, łączy ze sobą dwa węzły sieci tak, że widzą się one, jakby były bezpośrednio ze sobą połączone; protokół ten tworzy tunel, którym idą już właściwe dane w innym protokole (np. dalej IP i TCP).
	\end{itemize}
	\item \textbf{komutacja pakietów}
	\begin{itemize}
		\setlength\itemsep{1pt}
		\item \textbf{HDLC} bitowo zorientowany protokół warstwy łącza danych, używany do transmisji punktpunkt, jak i jeden-do-wielu, w sieciach z komutacją pakietów. Charakterystyczne jest to, że stacja może pracować w jednym z trzech trybów:
		
		\begin{itemize}
			\setlength\itemsep{1pt}
			\item stacja \textit{primary} – może inicjować transmisję ze stacją \textit{secondary}, kontroluje ją, może utrzymywać wiele sesji ze stacjami podrzędnymi
			\item stacja \textit{secondary} – wykonuje polecenia stacji nadrzędnej, utrzymuje tylko jedną sesję
			\item stacja uniwersalna – taki mix dwóch powyżej, może utrzymywać tylko jedną sesję
		\end{itemize}
		
		Podczas transmisji przesyłane są ramki o stałej strukturze:
		\begin{figure}[H]
			\begin{verbatim}
			         +--------+--------+--------+---------------+----------------+--------+
			         | FLAGA  |  ADRES | KONTR. |   INFORMACJE  |      FCS       |  FLAGA |
			         |    8   |    8   | 8 / 16 | (zmienna dł.) |       16       |    8   |
			         +--------+--------+--------+---------------+----------------+--------+
			\end{verbatim}
		\end{figure}
		
		\item \textbf{X.25} Protokół połączeniowy, łączycy węzły sieci rozległych, z komutacją pakietów: DCE \textit{Data Circuit-Terminating Equipment} – wewnętrzne węzły sieci rozległej, DTE \textit{Data Terminal Equipment} – system komputerowy użytkownika. Protokół X.25 definiuje tzw. połączenie wirtualne – czyli mamy kilka połączeń w jednym fizycznym kanale, które postrzegane są tak, jakby naprawdę istniało kilka kabli. Wśród połączeń wirtualnych wyróżnia się:
		
		\begin{itemize}
			\setlength\itemsep{1pt}
			\item PVC – czyli stałe połączenie wirtualne
			\item SVC – tymczasowe połączenie wirtualne
		\end{itemize}
		
		Protokół ten ma bardzo rozbudowane mechanizmy kontroli błędów (np. potwierdzanie każdej ramki), stąd nadal jest używany tam, gdzie wymaga się pewności transmisji kosztem prędkości (bankomaty, bankowość ogólnie, medycyna, monitoring waż- nych danych i zasobów). Aktualnie maksymalna prędkość uzyskiwana przy pomocy tego protokołu to 2 Mb/s.
		
		\item \textbf{Frame Relay} następca X.25, posiada prostsze mechanizmy kontroli poprawności, przez co transmisja jest o wiele szybsza niż w X.25. Reszta ogólnie tak, jak w X.25.
		
		Maksymalna prędkość transmisji to 45 Mb/s.
	\end{itemize}
	\item \textbf{komutacja komórek}
	\begin{itemize}
		\setlength\itemsep{1pt}
		\item \textbf{ATM} protokół połączeń w sieciach rozległych, wykorzystuje komutacje komórek. Dane przesyłane są w małych małych porcjach: nagłówek (5 bajtów) + dane (48 bajtów). Też występują tu kanały wirtualne oraz ścieżki wirtualne (orientacja połączeniowa). Kanał fizyczny dzielony jest na ścieżki wirtualne, a te dopiero na kanały wirtualne. Mechanizmy kontroli i sterowania przepływem muszą być realizowane po stronach systemów użytkowników końcowych. Maksymalna prędkość transmisji to 622 Mb/s.
	\end{itemize}
\end{itemize}
